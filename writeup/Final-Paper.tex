\documentclass{article}

% if you need to pass options to natbib, use, e.g.:
% \PassOptionsToPackage{numbers, compress}{natbib}
% before loading nips_2016
%
% to avoid loading the natbib package, add option nonatbib:
% \usepackage[nonatbib]{nips_2016}

\usepackage{nips_2016}

% to compile a camera-ready version, add the [final] option, e.g.:
% \usepackage[final]{nips_2016}

\usepackage[utf8]{inputenc} % allow utf-8 input
\usepackage[T1]{fontenc}    % use 8-bit T1 fonts
\usepackage{hyperref}       % hyperlinks
\usepackage{url}            % simple URL typesetting
\usepackage{booktabs}       % professional-quality tables
\usepackage{amsfonts}       % blackboard math symbols
\usepackage{nicefrac}       % compact symbols for 1/2, etc.
\usepackage{microtype}      % microtypography
\usepackage{color}
\title{Formatting instructions for NIPS 2016}

% The \author macro works with any number of authors. There are two
% commands used to separate the names and addresses of multiple
% authors: \And and \AND.
%
% Using \And between authors leaves it to LaTeX to determine where to
% break the lines. Using \AND forces a line break at that point. So,
% if LaTeX puts 3 of 4 authors names on the first line, and the last
% on the second line, try using \AND instead of \And before the third
% author name.

\author{
  David S.~Hippocampus\thanks{Use footnote for providing further
    information about author (webpage, alternative
    address)---\emph{not} for acknowledging funding agencies.} \\
  Department of Computer Science\\
  Cranberry-Lemon University\\
  Pittsburgh, PA 15213 \\
  \texttt{hippo@cs.cranberry-lemon.edu} \\
  %% examples of more authors
  %% \And
  %% Coauthor \\
  %% Affiliation \\
  %% Address \\
  %% \texttt{email} \\
  %% \AND
  %% Coauthor \\
  %% Affiliation \\
  %% Address \\
  %% \texttt{email} \\
  %% \And
  %% Coauthor \\
  %% Affiliation \\
  %% Address \\
  %% \texttt{email} \\
  %% \And
  %% Coauthor \\
  %% Affiliation \\
  %% Address \\
  %% \texttt{email} \\
}

\begin{document}
% \nipsfinalcopy is no longer used

\maketitle

\begin{abstract}
  The abstract paragraph should be indented \nicefrac{1}{2}~inch
  (3~picas) on both the left- and right-hand margins. Use 10~point
  type, with a vertical spacing (leading) of 11~points.  The word
  \textbf{Abstract} must be centered, bold, and in point size 12. Two
  line spaces precede the abstract. The abstract must be limited to
  one paragraph.
\end{abstract}

\section{Introduction}
Quantitative analysis of legislative data has the potential to provide new insights into how our government functions. Political scientists often focus on voting records of legislators on the suite of bills introduced during their term in congress. Indeed, {\color{red} even simple matrix factorizations and examination of prinicple components} of roll call data are able to uncover the political tendencies of individual representatives ({\color{red} figures}).

Another commonly utility of roll call vote data is to conduct {\itshape ideal point modeling}. Here, a congressman and a bill is presumed to lie in a latent `'ideoloigcal space,'' where the probability of a ``yay'' or ``nay'' response is a function of the bill's position and the congressman's position. The congressman's position is known as an `'ideal point'' because his or her utility decreases as a bill's position deviates from this point. One example of ideal point modeling in roll call data can be found in Gerrish and Blei 2011 where they assumed that ideal points lay in a one dimensionsal latent space; in this paper, we examine their results when we extend to higher dimensional ({\color{red} two?}) ideological spaces. \par
In addition to a senator's latent ideology, we furthur posit that the senators belong in latent communities. Using stochastic block modeling......


\section{The model}
\subsection{Ideal Point Model}

\subsection{Stochastic Block Model}


\section{Results}

\section{Discussion}




\section*{References}

\medskip

\small

[1] Gerrish, S.M.\ \& Blei, M.B. \ (2011) Predicting Legislative Roll Calls from Text. {\it Proceedings of the 28th International Conference on Machine Learning}

\appendix

\section{Variational updates}

\end{document}